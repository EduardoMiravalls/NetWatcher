\chapter{Conclusiones\label{cap:conclusiones}}

Este Trabajo de Fin de Grado ha consistido en el desarollo de una interfaz para la gestión de sondas de red de altas prestaciones.
Para ello, se ha analizado el problema, estudiado el estado del arte, definido de forma rigurosa y completa el proyecto a realizar, diseñado la solución, implementado la misma, llevado a cabo una verificación y validación del sistema y, por último, se ha ejecutado un mantenimiento correctivo.

La aplicación propuesta facilita la utilización de la sonda de red mediante una interfaz gráfica basada en tecnologías web.
Esta interfaz puede ser utilizada por cualquier usuario con conocimientos informáticos, sin ser necesario un conocimiento específico de la sonda en sí.
Se cumple así el objetivo de simplificar el empleo de la sonda seleccionada que anteriormente sólo se podía manejar por línea de comandos.
También agrupa otros aspectos relevantes de la captura y reproducción de tráfico web, como la gestión del almacenamiento de las \glspl{traza}, la velocidad de escritura en disco o la conversión entre formatos de \gls{traza}.
La implementación de la interfaz sigue un diseño \textit{responsive}, lo que hace que se pueda utilizar desde dispositivos móviles.
Se ha conseguido, como producto final, una interfaz elegante que permite gestionar todos los aspectos de la sonda considerados relevantes, presentando el estado e información adicional del sistema de forma visual, mediante gráficos y estadísticas.

Sin embargo, este proyecto no ha consistido únicamente en el desarrollo de una interfaz.
Se ha dividido el sistema en dos componentes, \gls{back-end} y \gls{front-end}, que se comunican entre sí y que pueden estar alojados en un servidor distinto cada uno.
Así, se ha diseñado e implementado un \gls{servicioweb} \gls{REST} en el \gls{back-end}, que formaliza el estado y funcionalidad de la \gls{FPGA}, añadiendo también control sobre otros aspectos relevantes mencionados anteriormente.
Para el \gls{front-end}, se ha creado un \gls{framework} propio, sobre el que se ha implementado la interfaz web.
Esta arquitectura de la aplicación cuenta con ciertas ventajas respecto a no separar físicamente el \gls{front-end} y el \gls{back-end}.
Al monitorizar el estado de la \gls{FPGA}, la interfaz estará disponible aunque el servidor que aloja la sonda no se encuentre operativo.
Por otra parte, al trasladar la interfaz a un servidor distinto, se libera parte de los recursos del servidor conectado a la sonda de red, afectando menos a su rendimiento.
Este proyecto, por tanto, proporciona una base sólida sobre la que se pueden incluir otras sondas de red sin partir desde cero, y siendo sólo necesario modificar una parte bien diferenciada del \gls{back-end}.

Para la implementación de la aplicación, se han utilizado diversos lenguajes de programación y herramientas, lo que constituye, sin duda, una valiosa experiencia para el futuro profesional del estudiante.
En el \gls{back-end} se ha empleado \textit{node.js}, una librería para construir aplicaciones de red en \textit{JavaScript} (tradicionalmente utilizado en el cliente, no en el servidor).
La codificación sobre esta librería se basa en la programación asíncrona, paradigma que ha sido enriquecedor aprender.
El \gls{front-end} se ha implementado con varios lenguajes de programación: \textit{PHP}, \textit{JavaScript}, \textit{HTML} y \textit{CSS}.
Se han empleado en este componente las librerías \textit{Bootstrap} y \textit{jQuery}, comunes en el desarrollo de páginas web.

Durante el desarrollo de este proyecto, se han podido poner en práctica conocimientos adquiridos en diversas asignaturas del Grado en Ingeniería Informática.
Por ejemplo, se ha seguido un proceso completo de desarrollo \textit{software}, aprendido en Ingeniería del Software.
Asignaturas como Arquitectura de Computadores y Redes de Comunicación han sido claves, al ser el trabajo base del que parte este proyecto una sonda de red (que a pesar de no haberse modificado, sí ha sido necesario entender su funcionamiento).
Conceptos como \gls{servicioweb}, \gls{API} \gls{REST} y algunos lenguajes de programación web fueron estudiados en Sistemas Informáticos I y II, lo que ha agilizado su adopción.

Por último, destacar que se ha liberado el código fuente del proyecto en la plataforma \textit{GitHub} (\url{https://github.com/JSidrach/NetWatcher}).
Esto no ha sido un simple gesto, ya que se ha invertido tiempo adicional en documentar todo el código de la aplicación en inglés.
Además, se han creado páginas \textit{wiki} dentro de la propia plataforma explicando el funcionamiento interno de la aplicación, junto con manuales de instalación, configuración, etc.
Se espera, con este esfuerzo, fomentar la participación de otros desarrolladores de la comunidad en la ampliación y mejora del sistema implementado (se dan algunas ideas en el capítulo~\ref{cap:lineasDeTrabajoFuturo}).
Liberar el proyecto ha consistido también en mejorar y poner a disposición de otros alumnos la plantilla en \textit{LaTeX} creada para esta memoria, que se encuentra disponible en otro repositorio público de \textit{GitHub} (\url{https://github.com/JSidrach/tfg-plantilla}).
A fecha de hoy, esta plantilla está siendo utilizada en varios trabajos más.
