% Resumen en inglés
\chapter*{Abstract}

\begin{abstractEn}

A network probe is a device capable of capturing or injecting network traffic.
This work is founded on a custom-made FPGA based probe.
So far, the only possibility to interact with this probe was through the command line, difficulting the management of the probe for people without any previous knowledge of it.
Another issue present in the probe's control was the existence of many aspects related to its operation that were being handled externally, such as storage, sorting and managing the traces or monitoring the system performance.

The proposed application facilitates the use of the network probe through a web-based graphical interface.
This interface allows the user to manage the probe without any specific knowledge of its inner workings.
In addition, it brings together other relevant aspects of the capture and injection of web traffic, such as trace storage, write speed of the disks or trace format conversion.
The interface implementation follows a responsive design and can be used on mobile devices with a similar user experience.
The final product provides an elegant interface to manage all the relevant aspects of the probe, which displays the state and additional information of the system visually, using graphs and statistics.

This project, however, has not exclusively consisted in the development of a graphical interface.
A base architecture, which formalizes the management of network probes from a web interface, has also been designed.
This proposal contemplates how to structure the components so that it is extensible to other network probes, not just the one selected.
With this goal in mind, the system has been divided into two components, back-end and front-end, which communicate with each other and that can be hosted on different servers.
Thereby, a REST Web Service has been designed and implemented for the back-end, which formalizes the status and functionality of the FPGA, adding also control over other important aspects mentioned above.
For the front-end, an ad hoc framework has been created, which has served as a starting point for the web interface.
It is intended that most of the proposed solution could be reused in similar projects, in which the management of the probes is done through command line.

\end{abstractEn}

% Palabras clave en inglés
\begin{keywordsEn}
web interface, responsive design, web service, API REST, capture and reproduction of network's traffic, FPGA.
\end{keywordsEn}

% Resumen en español
\chapter*{Resumen}

\begin{abstractEs}

Una sonda de red es un dispositivo capaz de capturar tráfico de red o de inyectarlo.
Este trabajo está basado en una sonda a medida sobre una FPGA.
Hasta ahora, la única posibilidad era interaccionar con esta sonda desde la línea de comandos, lo que dificultaba su gestión para personas sin conocimientos previos de la misma.
Otro problema presente en el control de la sonda era la existencia de numerosos aspectos relacionados con su funcionamiento que se manejaban de forma externa, tales como el almacenamiento, clasificación y gestión de las trazas, o la monitorización del rendimiento del sistema.

La aplicación propuesta facilita la utilización de la sonda de red mediante una interfaz gráfica basada en tecnologías web.
Esta interfaz permite gestionar la sonda sin tener un conocimiento específico de su funcionamiento interno.
También agrupa otros aspectos relevantes de la captura y reproducción de tráfico web, como el almacenamiento de las trazas, la velocidad de escritura en disco o la conversión entre formatos de traza.
La implementación de la interfaz sigue un diseño responsive, pudiendo utilizarse desde dispositivos móviles con una experiencia de usuario similar.
Se ha conseguido, como producto final, una interfaz elegante que permite gestionar todos los aspectos de la sonda considerados relevantes, presentando el estado e información adicional del sistema de forma visual, mediante gráficos y estadísticas.

Este proyecto no ha consistido únicamente en el desarrollo de una interfaz gráfica.
Se ha diseñado también una arquitectura base que formaliza la gestión de sondas de red desde una interfaz web.
Esta propuesta contempla cómo estructurar los componentes de forma que sea extensible a otras sondas de red, no sólo la seleccionada.
Con este objetivo, se ha dividido el sistema en dos componentes, back-end y front-end, que se comunican entre sí y que pueden estar alojados en un servidor distinto cada uno.
Así, se ha diseñado e implementado un Servicio Web REST en el back-end, que formaliza el estado y funcionalidad de la FPGA, añadiendo también control sobre otros aspectos relevantes mencionados anteriormente.
Para el front-end se ha creado un framework propio, que ha servido de punto de partida para la interfaz web.
Se pretende que gran parte de la solución propuesta sea reutilizable en proyectos similares, en los que el manejo de las sondas se realiza por línea de comandos.

\end{abstractEs}

% Palabras clave en español
\begin{keywordsEs}
interfaz web, diseño responsive, servicio web, API REST, captura y reproducción de tráfico de red, FPGA.
\end{keywordsEs}
