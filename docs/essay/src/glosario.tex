% Acrónimos

\newacronym{HTTP}{HTTP}{Hypertext Transfer Protocol}
\newacronym{URL}{URL}{Uniform Resource Locator}
\newacronym{FPGA}{FPGA}{Field-Programmable Gate Array}
\newacronym{IFG}{IFG}{InterFrame Gap (pausa temporal entre paquetes)}
\newacronym{RAID}{RAID}{Redundant Array of Independent Disks}
\newacronym{pcap}{pcap}{Packet capture (formato de traza, utilizado por programas como \textit{Wireshark} y \textit{tcpdump})}
\newacronym{API}{API}{Application Programming Interface (métodos públicos de una aplicación)}
\newacronym{JSON}{JSON}{JavaScript Object Notation}
\newacronym{PHP}{PHP}{PHP Hypertext Pre-processor}
\newacronym{AJAX}{AJAX}{Asynchronous JavaScript and XML}
\newacronym{HTML}{HTML}{HyperText Markup Language}
\newacronym{REST}{REST}{Representational State Transfer}

% Glosario

\newglossaryentry{bitstream}{name={bitstream},description={En este contexto se refiere al binario que configura el Hardware de la FPGA}}
\newglossaryentry{traza}{name={traza},plural={trazas},description={Archivo que contiene paquetes de red capturados}}
\newglossaryentry{simple}{name={simple},description={Formato de traza que acepta la FPGA utilizada}}
\newglossaryentry{servicioweb}{name={Servicio Web},description={Conjunto de métodos remotos accesibles a través de la red}}
\newglossaryentry{proxy}{name={proxy},description={Servidor que sirve de intermediario entre las peticiones de recursos que realiza un cliente a otro servidor}}
\newglossaryentry{framework}{name={framework},description={Entorno software que proporciona una funcionalidad base para facilitar la organización y desarrollo de aplicaciones similares}}
\newglossaryentry{back-end}{name={back-end},description={Componentes internos de la aplicación que procesan los datos provenientes del front-end}}
\newglossaryentry{front-end}{name={front-end},description={Interfaz, parte de la aplicación que interacciona directamente con el usuario}}
\newglossaryentry{script}{name={script},plural={scripts},description={Programa interpretado que se almacena en un archivo de texto plano}}
\newglossaryentry{log}{name={log},plural={logs},description={Fichero que contiene un registro de eventos}}
\newglossaryentry{codigoabierto}{name={código abierto},description={Software cuyo código fuente y otros derechos son publicados bajo una licencia que puede permitir a los usuarios utilizar, cambiar, mejorar el software y redistribuirlo}}
