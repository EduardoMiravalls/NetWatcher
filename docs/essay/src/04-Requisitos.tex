\chapter{Requisitos\label{cap:requisitos}}

En este capítulo se enumeran los requisitos de la aplicación a desarrollar. Para la elaboración de esta lista de requisitos se ha realizado un análisis sobre el problema planteado: diseñar un servicio que permita gestionar y monitorizar una \gls{FPGA} que captura y reproduce tráfico de red.
Este análisis se ha realizado principalmente mediante la consulta directa con los potenciales usuarios de la aplicación y la evaluación del estado del arte.

Se han agrupado los requistos en dos clases: funcionales y no funcionales.
Los primeros describen el comportamiento que tendrá la aplicación, y los segundos los atributos de calidad y restricciones de la misma.


\section{Requisitos Funcionales\label{sec:req:rf}}

Los requisitos funcionales que deberá cumplir la aplicación desarrollada son los siguientes:

\begin{enumerate}[align=left,before=\itshape,font=\normalfont,label=\bfseries RF. \arabic*]
  \item Se podrá conocer el estado actual de la \gls{FPGA} entre los posibles estados descritos en~\ref{fpga:estados}.
  \item Se podrá configurar la \gls{FPGA} para captura de tráfico de red.
  \item Una vez configurada la \gls{FPGA} para captura de tráfico de red, se le podrá ordenar que capture tráfico de red desde un puerto específico de la \gls{FPGA}.
  Este tráfico se irá guardando en una \gls{traza} en formato \gls{simple}, hasta llegar a un tamaño decidido por el usuario.
  \item Si existe una captura en curso, el se podrá parar dicha captura, borrándose la \gls{traza} asociada a la captura.
  \item Si existe una captura en curso, se podrán conocer los parámetros con los que se inició dicha captura, así como el tiempo que ha transcurrido desde el inicio y cuántos bytes se ha capturado hasta el momento.
  \item Se podrá configurar la \gls{FPGA} para la reproducción de una \gls{traza}.
  \item Una vez configurada la \gls{FPGA} para la reproducción de una \gls{traza}, se le podrá ordenar que reproduzca una \gls{traza} concreta en formato \gls{simple}.
  La reproducción se realizará con una una serie de parámetros dados por el usuario: máscara de puertos a los que dirigir la reproducción, \gls{IFG} asociado y reproducir en bucle o solo una vez.
  \item Si existe una reproducción de \gls{traza} en curso, se podrá parar dicha reproducción.
  \item Si existe una reproducción de \gls{traza} en curso, se podrán conocer los parámetros con los que se inició dicha reproducción, así como el tiempo que ha transcurrido desde el inicio y cuántos paquetes se han reproducido hasta el momento.
  \item Se podrá configurar y consultar en qué directorio se almacenan las \glspl{traza}.
  \item Se podrá conocer la lista de \glspl{traza} existentes, así como su tamaño, fecha y tipo (\gls{simple} o \gls{pcap}).
  \item Una traza en formato \gls{simple} podrá ser convertida a formato \gls{pcap}.
  \item Una traza en formato \gls{pcap} podrá ser convertida a formato \gls{simple}.
  \item Una \gls{traza} podrá ser renombrada.
  \item Una \gls{traza} podrá ser borrada.
  \item Se podrán conocer el espacio total y el espacio ocupado del sistema de archivos que contiene las \glspl{traza}.
  \item Si el sistema de archivos que contiene las \glspl{traza} es un \gls{RAID}, se podrá conocer la velocidad de escritura global del \gls{RAID}, así como la de cada disco que lo compone.
  \item Si el sistema de archivos que contiene las \glspl{traza} es un \gls{RAID}, se podrá formatear y recrear el \gls{RAID}.
\end{enumerate}


\section{Requisitos No Funcionales\label{sec:req:rnf}}

Los requisitos no funcionales que deberá cumplir la aplicación desarrollada son los siguientes:

\begin{enumerate}[align=left,before=\itshape,font=\normalfont,label=\bfseries RNF. \arabic*]
  \item La funcionalidad descrita en la sección~\ref{sec:req:rf} será accesible al usuario mediante una interfaz gráfica.
  \item Se podrán seleccionar dos idiomas para la interfaz gráfica: inglés y español.
  \item Se podrán seleccionar distintos temas (aspectos visuales) para la interfaz gráfica.
  \item La interfaz gráfica será una web adaptativa, de forma que se pueda visualizar en distintas resoluciones de pantalla, como las de ordenadores y móviles.
  \item La interfaz gráfica estará disponible aun cuando haya algún fallo en el servidor que aloja la \gls{FPGA}, e informará del error.
\end{enumerate}
