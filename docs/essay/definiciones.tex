% Clase del documento
\documentclass[a4paper,12pt,twoside,openright,titlepage]{book}

%
% Paquetes necesarios
%

% Símbolo del euro
\usepackage{eurosym}
% Codificación UTF8
\usepackage[utf8]{inputenc}
% Caracteres del español
\usepackage[spanish]{babel}
% Código, algoritmos, etc.
\usepackage{listings}
% Definición de colores
\usepackage{color}
% Extensión del paquete color
\usepackage[table,xcdraw]{xcolor}
% Márgenes
\usepackage{anysize}
% Cabecera y pie de página
\usepackage{fancyhdr}
% Estilo título capítulos
\usepackage{quotchap}
% Algoritmos (expresarlos mejor)
\usepackage{algorithmic}
% Títulos de secciones
\usepackage{titlesec}
% Fórmulas matemáticas
\usepackage[cmex10]{amsmath}
% Enumeraciones
\usepackage{enumerate}
% Páginas en blanco
\usepackage{emptypage}
% Separación entre cajas
\usepackage{float}
% Imágenes
\usepackage[pdftex]{graphicx}
% Mejora de las tablas
\usepackage{array}
% Mejora de los símbolos matemáticos
\usepackage{mdwmath}
% Separar figuras en subfiguras
\usepackage[caption=false,font=footnotesize]{subfig}
% Incluir pdfs externos
\usepackage{pdfpages}
% Mejoras sobre las cajas
\usepackage{fancybox}
% Apéndices
\usepackage{appendix}
% Marcadores (para el pdf)
\usepackage{bookmark}
% Estilo de enumeraciones
\usepackage{enumitem}
% Espacio entre líneas y párrafos
\usepackage{setspace}
% Glosario/Acrónimos
\usepackage[acronym]{glossaries}
% Fuentes
\usepackage[T1]{fontenc}
% Bibliografía
\usepackage[sorting=none,natbib=true,backend=bibtex]{biblatex}

% Enlaces
\hypersetup{hidelinks,pageanchor=false,colorlinks,citecolor=Fuchsia,urlcolor=black,linkcolor=Cerulean}

% Fix warning
\setlength{\headheight}{16pt}

% Euro (€)
\DeclareUnicodeCharacter{20AC}{\euro}

% Texto referencias
\addto{\captionsspanish}{\renewcommand{\bibname}{Referencias}}

% Inclusión de gráficos
\graphicspath{{./graphics/}}

% Extensiones de gráficos
\DeclareGraphicsExtensions{.pdf,.jpeg,.jpg,.png}

% Definiciones de colores (para hidelinks)
\definecolor{LightCyan}{rgb}{0,0,0}
\definecolor{Cerulean}{rgb}{0,0,0}
\definecolor{Fuchsia}{rgb}{0,0,0}

% Keywords (español e inglés)
\def\keywordsEn{\vspace{.5em}
{\textbf{\textit{Key words ---}}\,\relax%
}}
\def\endkeywordsEn{\par}

\def\keywordsEs{\vspace{.5em}
{\textbf{\textit{Palabras clave ---}}\,\relax%
}}
\def\endkeywordsEs{\par}


% Abstract (español e inglés)
\def\abstractEs{\vspace{.5em}}
\def\endabstractEs{\par}

\def\abstractEn{\vspace{.5em}}
\def\endabstractEn{\par}

% Estilo páginas de capítulos
\fancypagestyle{plain}{
\fancyhf{}
\fancyfoot[CO]{\footnotesize\emph{\nombretrabajo}}
\fancyfoot[RO]{\thepage}
\renewcommand{\footrulewidth}{.6pt}
\renewcommand{\headrulewidth}{0.0pt}
}

% Estilo resto de páginas
\pagestyle{fancy}

% Estilo páginas impares
\fancyfoot[CO]{\footnotesize\emph{\nombretrabajo}}
\fancyfoot[RO]{\thepage}
\rhead[]{\leftmark}

% Estilo páginas pares
\fancyfoot[CE]{\emph{\pieparcen}}
\fancyfoot[LE]{\thepage}
\fancyfoot[RE]{\pieparizq}
\lhead[\leftmark]{}

% Guía del pie de página
\renewcommand{\footrulewidth}{.6pt}

% Nombre de los bloques de código
\renewcommand{\lstlistingname}{Código}

% Estilo de los lstlistings
\lstset{
    frame=tb,
    breaklines=true,
    postbreak=\raisebox{0ex}[0ex][0ex]{\ensuremath{\color{gray}\hookrightarrow\space}}
}

% Definiciones de funciones para los títulos
\newlength\salto
\setlength{\salto}{3.5ex plus 1ex minus .2ex}
\newlength\resalto
\setlength{\resalto}{2.3ex plus.2ex}

% Estilo de los acrónimos
\renewcommand{\acronymname}{Glosario}
\renewcommand{\glossaryname}{Glosario}
\pretolerance=2000
\tolerance=3000
\renewcommand{\glsnamefont}[1]{\makefirstuc{#1}}

% Texto índice de tablas
\addto\captionsspanish{
\def\tablename{Tabla}
\def\listtablename{\'Indice de tablas}
}

% Traducir appendix/appendices
\renewcommand\appendixtocname{Apéndices}
\renewcommand\appendixpagename{Apéndices}

% Comando code (lstlisting sin cambio de página)
\lstnewenvironment{code}[1][]%
  { \noindent\minipage{0.935\linewidth}\medskip
    \vspace{5mm}
    \lstset{basicstyle=\ttfamily\footnotesize,#1}}
  {\endminipage}

% Estilo JSON en listing
\colorlet{punct}{red!60!black}
\definecolor{background}{HTML}{EEEEEE}
\definecolor{delim}{RGB}{20,105,176}
\colorlet{numb}{magenta!60!black}
\lstdefinelanguage{json}{
    basicstyle=\normalfont\ttfamily,
    stepnumber=1,
    numbersep=8pt,
    showstringspaces=false,
    breaklines=true,
    backgroundcolor=\color{background},
    literate=
     *{0}{{{\color{numb}0}}}{1}
      {1}{{{\color{numb}1}}}{1}
      {2}{{{\color{numb}2}}}{1}
      {3}{{{\color{numb}3}}}{1}
      {4}{{{\color{numb}4}}}{1}
      {5}{{{\color{numb}5}}}{1}
      {6}{{{\color{numb}6}}}{1}
      {7}{{{\color{numb}7}}}{1}
      {8}{{{\color{numb}8}}}{1}
      {9}{{{\color{numb}9}}}{1}
      {:}{{{\color{punct}{:}}}}{1}
      {,}{{{\color{punct}{,}}}}{1}
      {\{}{{{\color{delim}{\{}}}}{1}
      {\}}{{{\color{delim}{\}}}}}{1}
      {[}{{{\color{delim}{[}}}}{1}
      {]}{{{\color{delim}{]}}}}{1},
}
