\section{Conclusiones}

\begin{frame}{Conclusiones}
  \begin{itemize}
    \item Gestión del sistema completo mediante una interfaz web
    \begin{itemize}
      \item Control de la FPGA de forma intuitiva
    \end{itemize}
    \item Trabajo extensible a otras sondas de red
    \begin{itemize}
      \item Arquitectura base
      \item Módulos independientes del modelo de sonda
      \begin{itemize}
        \item Clasificación y conversión de trazas
        \item Almacenamiento
        \item Velocidad del disco
        \item Estado global del sistema
      \end{itemize}
    \end{itemize}
    \item Se registra el uso tanto del Back-End como del Front-End
    \item Se cumplen todos los objetivos propuestos
  \end{itemize}
\end{frame}

\begin{frame}{Contribuciones}
  \begin{itemize}
    \item Proyecto europeo de federación Fed4FIRE
    \begin{itemize}
      \item Desarrollado de forma paralela al TFG
      \item Back-End utilizado en la integración del testbed de la FPGA
      \begin{itemize}
        \item Permite exponer la funcionalidad de la sonda de red sin que el usuario tenga acceso al servidor al que está conectada
      \end{itemize}
    \end{itemize}
    \item The Open Source Network Tester
    \begin{itemize}
      \item Posible incorporación de la interfaz desarrollada
    \end{itemize}
    \item Proyecto disponible en GitHub
    \begin{itemize}
      \item Liberado bajo licencia MIT
      \item Documentación extensiva
    \end{itemize}
  \end{itemize}
\end{frame}
